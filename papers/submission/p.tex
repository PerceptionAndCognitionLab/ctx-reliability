\documentclass[english,man]{apa6}

\usepackage{amssymb,amsmath}
\usepackage{ifxetex,ifluatex}
\usepackage{fixltx2e} % provides \textsubscript
\ifnum 0\ifxetex 1\fi\ifluatex 1\fi=0 % if pdftex
  \usepackage[T1]{fontenc}
  \usepackage[utf8]{inputenc}
\else % if luatex or xelatex
  \ifxetex
    \usepackage{mathspec}
    \usepackage{xltxtra,xunicode}
  \else
    \usepackage{fontspec}
  \fi
  \defaultfontfeatures{Mapping=tex-text,Scale=MatchLowercase}
  \newcommand{\euro}{€}
\fi
% use upquote if available, for straight quotes in verbatim environments
\IfFileExists{upquote.sty}{\usepackage{upquote}}{}
% use microtype if available
\IfFileExists{microtype.sty}{\usepackage{microtype}}{}

% Table formatting
\usepackage{longtable, booktabs}
\usepackage{lscape}
% \usepackage[counterclockwise]{rotating}   % Landscape page setup for large tables
\usepackage{multirow}		% Table styling
\usepackage{tabularx}		% Control Column width
\usepackage[flushleft]{threeparttable}	% Allows for three part tables with a specified notes section
\usepackage{threeparttablex}            % Lets threeparttable work with longtable

% Create new environments so endfloat can handle them
% \newenvironment{ltable}
%   {\begin{landscape}\begin{center}\begin{threeparttable}}
%   {\end{threeparttable}\end{center}\end{landscape}}

\newenvironment{lltable}
  {\begin{landscape}\begin{center}\begin{ThreePartTable}}
  {\end{ThreePartTable}\end{center}\end{landscape}}

  \usepackage{ifthen} % Only add declarations when endfloat package is loaded
  \ifthenelse{\equal{\string man}{\string man}}{%
   \DeclareDelayedFloatFlavor{ThreePartTable}{table} % Make endfloat play with longtable
   % \DeclareDelayedFloatFlavor{ltable}{table} % Make endfloat play with lscape
   \DeclareDelayedFloatFlavor{lltable}{table} % Make endfloat play with lscape & longtable
  }{}%



% The following enables adjusting longtable caption width to table width
% Solution found at http://golatex.de/longtable-mit-caption-so-breit-wie-die-tabelle-t15767.html
\makeatletter
\newcommand\LastLTentrywidth{1em}
\newlength\longtablewidth
\setlength{\longtablewidth}{1in}
\newcommand\getlongtablewidth{%
 \begingroup
  \ifcsname LT@\roman{LT@tables}\endcsname
  \global\longtablewidth=0pt
  \renewcommand\LT@entry[2]{\global\advance\longtablewidth by ##2\relax\gdef\LastLTentrywidth{##2}}%
  \@nameuse{LT@\roman{LT@tables}}%
  \fi
\endgroup}


  \usepackage{graphicx}
  \makeatletter
  \def\maxwidth{\ifdim\Gin@nat@width>\linewidth\linewidth\else\Gin@nat@width\fi}
  \def\maxheight{\ifdim\Gin@nat@height>\textheight\textheight\else\Gin@nat@height\fi}
  \makeatother
  % Scale images if necessary, so that they will not overflow the page
  % margins by default, and it is still possible to overwrite the defaults
  % using explicit options in \includegraphics[width, height, ...]{}
  \setkeys{Gin}{width=\maxwidth,height=\maxheight,keepaspectratio}
\ifxetex
  \usepackage[setpagesize=false, % page size defined by xetex
              unicode=false, % unicode breaks when used with xetex
              xetex]{hyperref}
\else
  \usepackage[unicode=true]{hyperref}
\fi
\hypersetup{breaklinks=true,
            pdfauthor={},
            pdftitle={A Psychometrics of Individual Differences in Experimental Tasks},
            colorlinks=true,
            citecolor=blue,
            urlcolor=blue,
            linkcolor=black,
            pdfborder={0 0 0}}
\urlstyle{same}  % don't use monospace font for urls

\setlength{\parindent}{0pt}
%\setlength{\parskip}{0pt plus 0pt minus 0pt}

\setlength{\emergencystretch}{3em}  % prevent overfull lines

\ifxetex
  \usepackage{polyglossia}
  \setmainlanguage{}
\else
  \usepackage[english]{babel}
\fi

% Manuscript styling
\captionsetup{font=singlespacing,justification=justified}
\usepackage{csquotes}
\usepackage{upgreek}



\usepackage{tikz} % Variable definition to generate author note

% fix for \tightlist problem in pandoc 1.14
\providecommand{\tightlist}{%
  \setlength{\itemsep}{0pt}\setlength{\parskip}{0pt}}

% Essential manuscript parts
  \title{A Psychometrics of Individual Differences in Experimental Tasks}

  \shorttitle{Psychometrics of Tasks}


  \author{Jeffrey N. Rouder\textsuperscript{1}~\& Julia M. Haaf\textsuperscript{2}}

  % \def\affdep{{"", ""}}%
  % \def\affcity{{"", ""}}%

  \affiliation{
    \vspace{0.5cm}
          \textsuperscript{1} University of California, Irvine\\
          \textsuperscript{2} University of Missouri  }

  \authornote{
    We are indebted to Craig Hedge for making data available, working with
    us to insure we understand them, and providing comments on this work.
    The data and analysis code for this project may be found at
    \url{https://github.com/PerceptionAndCognitionLab/ctx-reliability}
    
    Correspondence concerning this article should be addressed to Jeffrey N.
    Rouder, . E-mail:
    \href{mailto:jrouder@uci.edu}{\nolinkurl{jrouder@uci.edu}}
  }

  \note{Version 2, 7/2018}

  \abstract{In modern individual-difference studies, researchers often correlate
performance on various tasks to uncover common latent processes. Yet, in
some sense, the results have been disappointing as correlations among
tasks that seemingly have processes in common are often low. A pressing
question then is whether these attenuated correlations reflect
statistical considerations, such as a lack of individual variability on
tasks, or substantive considerations, such as that inhibition in
different tasks is not a unified concept. One problem in addressing this
question is that researchers aggregate performance across trials to
tally individual-by-task scores, and the covariation of these scores is
subsequently studied much as it would be with classical test theory. It
is tempting to think that aggregation here is fine and everything comes
out in the wash, but as shown here, it greately attenuates measures of
effect size and correlation. We propose an alternative psychometrics of
task performance that is based on accounting for trial-by-trial
variability along with the covariation of individuals' performance
across tasks. The implementation is through common hierarchical models,
and this treatment rescues classical concepts of effect size,
reliability, and correlation for studying individual differences with
experimental tasks. Using recent data from Hedge et al. (2018) we show
that there is Bayes-factor support for a lack of correlation between the
Stroop and flanker task. This support for a lack of correlation
indicates a psychologically relevant result---Stroop and flanker
inhibition are seemingly unrelated, contradicting unified concepts of
inhibition.}
  \keywords{Individual Differences, Inhibition, Reliability, Hierarchical Models,
Bayesian Inference \\

    
  }




  \usepackage{bm}
  \usepackage{amsmath}
  \usepackage{setspace}
  \usepackage{pcl}
  \usepackage{makecell}

\usepackage{amsthm}
\newtheorem{theorem}{Theorem}[section]
\newtheorem{lemma}{Lemma}[section]
\theoremstyle{definition}
\newtheorem{definition}{Definition}[section]
\newtheorem{corollary}{Corollary}[section]
\newtheorem{proposition}{Proposition}[section]
\theoremstyle{definition}
\newtheorem{example}{Example}[section]
\theoremstyle{definition}
\newtheorem{exercise}{Exercise}[section]
\theoremstyle{remark}
\newtheorem*{remark}{Remark}
\newtheorem*{solution}{Solution}
\begin{document}

\maketitle

\setcounter{secnumdepth}{0}



In individual-differences studies, a number of variables are measured
for each individual. The goal is to decompose the covariation among
these variables into a lower-dimensional, theoretically-relevant
structure (Bollen, 1989; Skrondal \& Rabe-Hesketh, 2004). Critical in
this endeavor is understanding the psychometric properties of the
measurements. Broadly speaking, variables used in individual-difference
studies come from the following three classes: The first is the class of
rather natural and easy-to-measure variables such age, weight, and
gender. The second is the class of \emph{instruments} such as
personality and psychopathology instruments. Instruments have a fixed
battery of questions and a fixed scoring algorithm. Most instruments
have been benchmarked, and their reliability has been well established.
The final class of variables is performance on experimental tasks. These
experimental tasks are often used to assess cognitive abilities in
memory, attention, and perception.

On the face of it, individual-difference researchers should be confident
about using scores from experimental tasks for the following reasons:
First, many of these tasks are robust in that the effects are easy to
obtain in a variety of circumstances. Take, for example, the Stroop
task, which may be used as a measure of inhibition. The Stroop effect is
so robust, it is considered universal (Haaf \& Rouder, 2017; MacLeod,
1991). Second, many of these tasks have been designed to isolate
specific cognitive processes. The Stroop task, for example, requires
participants to inhibit the prepotent process of reading. Third, because
these tasks are laboratory based and center on experimenter-controlled
manipulations, they often have a high degree of internal validity.
Fourth, because these tasks are used so often, there is usually a large
literature about them to guide implementation and interpretation. It is
no wonder task-based measures have become popular in the study of
individual differences.

Yet, there has been a wrinkle in the setup. As it turns out, these
task-based measures designed to measure the same latent concepts do not
correlate heavily with one another. The best example of these wrinkles
is perhaps in the individual-differences study of inhibition tasks
(Friedman \& Miyake, 2004; Ito et al., 2015; Pettigrew \& Martin, 2014;
Rey-Mermet, Gade, \& Oberauer, 2018; Stahl et al., 2014). In these
large-scale individual-differences studies, researchers correlated
scores in a variety of tasks that require inhibitory control. An example
of two such tasks are the Stroop task (Stroop, 1935) and the flanker
task (Eriksen \& Eriksen, 1974). Correlations among inhibition measures
are notoriously low; most do not exceed .2 in value. The Stroop and
flanker measures, in particular, seemingly do not correlate at all
(Hedge, Powell, \& Sumner, 2018; Rey-Mermet et al., 2018; Stahl et al.,
2014). These low correlations are not limited to measures of inhibition.
Ito et al. (2015) considered several implicit attitude tasks used for
measuring implicit bias. Here again, there is surprisingly little
correlation among bias measures that purportedly measure the same
concept.

Why correlations are so low among these task-based measures is a mystery
(Hedge et al., 2018; Rey-Mermet et al., 2018). After all, the effects
are robust, easy-to-obtain, and seemingly measure the same basic
concepts. There are perhaps two leading explanations: One is that the
problem is mostly substantive. The presence of low correlations in
large-sampled studies may imply that the tasks are truly measuring
different sources of variation. In the inhibition case, the ubiquity of
low correlations in large-sample studies has led Rey-Mermet et al.
(2018) to make a substantive claim that the psychological concept of
inhibition is not unified at all. The second explanation is that the
lack of correlation is mostly methodological. High correlations may only
be obtained if the tasks are highly reliable, and low correlations are
not interpretable without high reliability. Hedge and colleagues
document a range of test-retest reliabilities across tasks, with most
tasks having reliabilities below .7 in value, and some tasks having
reliabilities below .4 in value. In effect, the problem is that people
simply don't vary enough in some tasks given the resolution of the data
to document correlations.

Researchers typically use classical test theory, at least implicitly, to
inform methodology when studying individual differences. They make this
choice when they tally up an individual's data into a performance score.
For example, in the Stroop task, we may score each person's performance
by the mean difference in response times between the incongruent and
congruent trials. Because there is a single performance score per person
per task, tasks are very much treated as instruments. When trials are
aggregated into a single score, the instrument (or experiment) is a
\emph{test} and classical test theory serves as the analytic framework.

The unit of analysis in classical test theory is the test, or in our
case, the experimental task. Theoretically important quantities, say
effect size within a task and correlations across tasks, for example,
are assigned to the task itself and not to a particular sample or a
particular sample size. For instance, when a test developer states that
an instrument has test-retest correlation of .80, that value holds as a
population truth for all samples. We may see more noise if we
re-estimate that value in smaller samples, but, on whole, the underlying
population value of an instrument does not depend on the researcher's
sample size. We call this desirable property \emph{portability.}

The main problem with using classical test theory for experimental
tasks, however, is that portability is violated. Concepts of effect
sizes within a task and correlations across tasks are critical functions
of sample sizes (Green et al., 2016). As a result, different researchers
will estimate different truths if they have different sample sizes.

In the next section, we show how dramatically portability is violated in
practice, and how much these violations affect the interpretability of
classical statistics. These failures serve as motivation for a call to
use hierarchical linear models that model trial-by-trial variation as
well as variation across individuals. We develop these models, and then
apply them to address the low-correlation mystery between flanker and
Stroop tasks.

\section{The Dramatic Failure of
Portability}\label{the-dramatic-failure-of-portability}

In classical test theory, an \emph{instrument}, say a depression
inventory, is a fixed unit. It has a fixed number of questions that are
often given in a specific order. When we speak of portability, we speak
of porting characteristics of this fixed unit to other settings, usually
to other sample sizes in similar populations. In experimental tasks, we
have several different sample sizes. One is the number of individuals in
the sample; another is the number of trials each individual completed
within the various conditions in the task. We wish to gain portability
across both of these sample-size dimensions. For example, suppose one
team runs 50 people each observing 50 trials in each condition and
another team runs 100 people each observing 25 trials in each condition.
If measures of reliability and effect size are to be considered a
property of a task, then these properties should be relatively stable
across both teams. Moreover, if performance correlated with other
variables, portability would mean that the correlation is stable as
well.

But standard psychometric measures that are portable for instruments
fail dramatically on tasks. To show this failure, we reanalyze data from
a few tasks from Hedge et al. (2018). Hedge et al. compiled an
impressively large data set. They asked individuals to perform a large
number of trials on several tasks. For example, in their Stroop task,
participants completed a total of 1440 trials each. The usual number for
individual-differences studies is on the order of 10s or 100s of trials
each. Moreover, Hedge et al. explicitly set up their design to measure
test-retest reliability. Individuals performed 720 trials in the first
session, and then, three-weeks later, performed the remaining 720 trials
in a second session.\footnote{Hedge ran 144 trials per block with 5
  blocks per session for a total of 10 blocks. These 144 trials were
  divided into three conditions: congruent, neutral, and incongruent. We
  do not use the neutral condition in our analysis. As a result, there
  are 48 trials per individual per condition per block, or 480 trials
  per individual per condition in the experiment. Once we discarded
  errors, RTs that were too fast or slow, and warm-up trials, there were
  on average 42 trials per individual per condition per block. Our
  discard parameters are documented in the markdown version of this
  document.} Here, we use the Hedge et al. data set to assess how
conventional measures of effect size and reliability are dependent on
the number of trials per condition.

The following notation is helpful for specifying the conventional
approach to effect size and reliability. Let \(\bar{Y}_{ijk}\) be the
mean response time for the \(ith\) individual in the \(j\)th session
(either the first or second) in the \(k\)th condition. For the Stroop
task, the conditions are congruent (\(k=1\)) or incongruent (\(k=2\)).
Effects in a session are denoted \(d_{ij}\). These are the differences
between mean incongruent and congruent response times:

\begin{equation} \label{samp.eff}
d_{ij}=\bar{Y}_{ij2}-\bar{Y}_{ij1}. 
\end{equation}

Two quantities of interest can be calculated from these individual
effects: The effect size and the test-retest reliability. To calculate
the effect size, we first average the session-by-individual effects
across sessions to obtain individuals' effects denoted \(\bar{d}_{i}\).
The effect size is just the ratio of the grand mean of the individual
effects to the standard deviation of these effects, i.e.,
es\(=\)mean(\(\bar{d}_{i}\))\(/\)sd(\(\bar{d}_{i}\)). The test-retest
reliability is the correlation of session-by-individual effect scores
for the first session (\(j=1\)) to the second session (\(j=2\)).

Figure~\ref{fig:portability} shows a lack of portability in these two
measures. We randomly selected different subsets of the Hedge et al.'s
data and computed effect sizes and test-retest reliabilities. We varied
the number of trials per condition per individual from 20 to 400, and
for each level, we collected 100 different randomly-chosen subsets. For
each subset, we computed effect size and reliability, and plotted are
the means across these 100 subsets. The line denoted \enquote{S} shows
the case for the Stroop task, and both effect size and reliability
measures increase appreciably with the number of trials per condition
per individual. The line denoted \enquote{F} is from Hedge et al.'s
flanker task, and the results are similar. In fact, these increases with
the number of replicates are a general property of tasks. Measures that
are treated as properties of the task are critically tied to sample
sizes. In summary, classical measures are portable when applied to
instruments because the numbers of items are fixed. They are importable
when applied to task performance where the number of trials per
individual will vary across studies.

\begin{figure}[htbp]
\centering
\includegraphics{p_files/figure-latex/portability-1.pdf}
\caption{\label{fig:portability}The effect of the
number-of-trials-per-individual on sample effect sizes and sample
reliability for Stroop and Flanker data from Hedge et al. (2018). Each
point is the mean over 100 different samples at the indicated sample
size.}
\end{figure}

The critical and outsized role of replicates is seemingly
under-appreciated in practice. Many researchers are quick to highlight
the numbers of individuals in studies. You may find this number
repeatedly in tables, abstracts, method sections, and sometimes in
general discussions. Researchers do not, however, highlight the number
of replicates per condition per individual. These numbers rarely appear
in abstracts or tables; in fact, it usually takes careful hunting to
find them in the method section if they are there at all. And
researchers are far less likely to discuss the numbers of replicates in
interpreting their results. And yet, as shown above, this number of
replicates is absolutely critical in understanding classical results.

\section{A Hierarchical Model}\label{a-hierarchical-model}

\subsection{Gaining Portability in The Limit of Infinite
Trials}\label{gaining-portability-in-the-limit-of-infinite-trials}

In this paper, we strive for portability---the meaning of correlation
and effect size should be independent of the number of trials in
component tasks. Portability is largely solved in classical test theory
by using standardized instruments, say intelligence tests of fixed
length. Yet, we think it is unreasonable to expect the same of
experimentalists. Experimentalists will invariably use different number
of trials depending on many factors including their access to subject
pools, their overall goals, and the number of tasks and conditions in
their experiments. We suspect recommending a standard number of trials
per experiment will be unsuccessful.

The solution to portability that we develop here is the implementation
of hierarchical statistical models to \enquote{correct} measurements for
trial-by-trial noise. In our case, where theoretical targets are sizes
of effects within tasks and correlations of these effects across tasks,
noise from finite trials serves as a nuisance. Researchers are most
interested in measurement of individuals' \emph{true} scores, that is,
the measurement of individuals' true flanker-effect and Stroop-effect
scores, and the true correlation between them. While individuals perform
finite trials, we may use statistical models to query what may
reasonably happen in the limit. And estimates of effect sizes and
correlations in the large-sample limit become portable estimates of
effect sizes and correlations.

Trial-noise-free estimates may be made with a hierarchical linear model
that accounts for trial-by-trial variation and individual variation
simultaneously. The application of hierarchical models to trial-by-trial
performance data in experimental tasks is not new. Previous applications
in individual-differences research include Schmiedek, Oberauer, Wilhelm,
Süß, and Wittmann (2007) and Voelkle, Brose, Schmiedek, and Lindenberger
(2014). Moreover, trial-by-trial hierarchical modeling is well known in
cognitive psychology (Lee \& Webb, 2005; Rouder \& Lu, 2005) and
linguistics (Baayen, Tweedie, \& Schreuder, 2002). That said, to our
knowledge, using hierarchical models to address the low-correlation
mystery and to make classical test-theory concepts portable in
experimental settings is novel.

Practitioners of classical test theory sometimes account for
trial-by-trial noise by treating it as measurement error. The goal is to
calculate this measurement error and then subtract it out. A good
example comes from Spearman (1904a), who describes how correlations may
be corrected for attenuation from measurement error (see
\url{https://en.wikipedia.org/wiki/Correction_for_attenuation}). We find
that our hierarchical model estimates agree with these corrections, but
the models offer a deeper understanding of the statistical and
theoretical dynamics at play.

In the next section we provide a fairly formal exposition of the model.
We use an equation-based rather than graphical presentation of the model
because it is from the equations that the main results flow. Using the
equations, we derive an expression for the sample effects and show how
the values are attenuated by trial variability. In the following
section, we do the same for sample correlation between two tasks, and
again show how the values are attenuated by trial noise. We show how and
why the hierarchical model provides for unattenuated estimates.

\subsection{Model specification}\label{model-specification}

Understanding what the model is and how it works relies on some basic
notion. Let \(I\) be the number of people, \(J\) be the number of tasks,
\(K\) be the number of conditions, and \(L\) be the number of trials per
person per task per condition, or the number of replicates. Subscripts
are used to denote individuals, tasks, conditions, and replicates. Let
\(Y_{ijk\ell}\) denote the \(\ell\)th observation,
\(\ell = 1, \ldots, L\) for the \(i\)th individual, \(i = 1, \ldots, I\)
in the \(j\)th task, \(j = 1, \ldots, J\) and \(k\)th condition,
\(k = 1, 2\). Observations are usually performance variables, and in our
case, and for concreteness, they can be response times on trials. For
now, we model response times in just one task, and in this case, the
subscript \(j\) may be omitted. Consider a trial-level base model: \[
Y_{ik\ell} \sim \mbox{Normal}(\mu_{ik},\sigma^2),
\] where \(\mu_{ik}\) is the true mean response time of the \(i\)th
person in the \(k\)th condition and \(\sigma^2\) is the true
trial-by-trial variability.

It is important to differentiate between true parameter values like
\(\mu_{ik}\) and their sample estimates.

We develop this model for the Stroop task. In this task, the key
contrast is between the congruent (\(k=1\)) and incongruent (\(k=2\))
conditions. This contrast is embedded in a model where each individual
has an average speed effect, denoted \(\alpha_i\), and a Stroop effect,
denoted \(\theta_i\): \[
Y_{ik\ell} \sim \mbox{Normal}(\alpha_{i}+x_k\theta_{i},\sigma^2),
\] where \(x_1=-1/2\) for the congruent condition and \(x_2=1/2\) for
the incongruent condition.

The goal then is to study \(\theta_i\), the \(i\)th person's Stroop
effect. In modern mixed models, individual's parameters \(\alpha_i\) and
\(\theta_i\) are considered latent traits for the \(i\)th person, and
are modeled as random effects: \[
\begin{aligned}
\alpha_i &\sim \mbox{Normal}(\mu_\alpha,\sigma^2_\alpha),\\
\theta_i &\sim \mbox{Normal}(\mu_\theta,\sigma^2_\theta),
\end{aligned}
\] where \(\mu_\alpha\) and \(\mu_\theta\) are population means and
\(\sigma^2_\alpha\) and \(\sigma^2_\theta\) are population variances.

\subsection{Hierarchical Regularization and the Portability of Effect
Size}\label{hierarchical-regularization-and-the-portability-of-effect-size}

Even before applying the model, a bit of analysis shows the flaws in
conventional analysis and the expected improvements from hierarchical
modeling. The conventional analysis of the Stroop task centers on sample
means aggregated over trials. The critical quantity is
\(d_i=\bar{Y}_{i2}-\bar{Y}_{i1}\), the observed Stroop effect for the
\(i\)th individual. It is helpful to express the distribution of sample
effects, \(d_i\) in model parameters:\footnote{To derive this
  distribution, note that
  \(\bar{Y}_{i1}|\alpha_i,\theta_i \sim \mbox{Normal}(\alpha_{i}-\theta_{i}/2,\sigma^2/L)\)
  and
  \(\bar{Y}_{i2}|\alpha_i,\theta_i \sim \mbox{Normal}(\alpha_{i}+\theta_{i}/2,\sigma^2/L)\).
  These random variables are \emph{conditionally independent}. Hence,
  the mean and variance come about from summing. Therefore
  \((\bar{Y}_{i2}-\bar{Y}_{i1})|\alpha_i,\theta_i \sim \mbox{Normal}(\theta_i,2\sigma^2/L)\).
  Marginalizing over \(\theta_i\) yields the distribution.} \[
d_i \sim \mbox{Normal}(\mu_\theta, 2\sigma^2/L + \sigma^2_\theta),
\] where \(L\) is the number of trials per individual per condition. The
term \(2\sigma^2/L\) corresponds to variability from trial-by-trial
noise and \(\sigma^2_\theta\) corresponds to variability of individuals'
effects. A classical effect size measure,
\(\mbox{mean}(d)/\mbox{sd}(d)\), therefore estimates
\(\mu_\theta/\sqrt{2\sigma^2/L + \sigma^2_\theta}\). The problem is the
inclusion of \(2\sigma^2/L\), the nuisance trial-by-trial variation.
This included nuisance trial variation results in individual effect
estimates that are too variable and in effect size measures that are too
small. More importantly, portability is violated as there is an explicit
dependence on \(L\), the number of trials. In the model-based approach,
the critical parameter is \(\theta_i\), and its distribution is: \[
\theta_i \sim \mbox{Normal}(\mu_\theta,\sigma^2_\theta),
\] The effect size calculated from these individual effects is an
estimate of \(\mu_\theta/\sigma_\theta\), a portable quantity.

It is now clear what aggregation and the hierarchical model do.
Aggregation is a way of getting at the true values, but only in the
large-trial limit (as \(L\to\infty\)). In this limit, the term
\(2\sigma^2/L\) becomes vanishingly small. The hierarchical model
provides an estimate of the same quantity, but it estimates the correct
quantity even with finite trials! By using it, results are portable to
designs with varying numbers of individuals and trials per individual.
These results are estimates of underlying, theoretically useful
properties of tasks.

How does \(d_i\) compare to model-based estimates of effects,
\(\theta_i\)? The hierarchical model is an example of a linear mixed
model, and as such, may be implemented in a variety of packages
including \texttt{SAS\ PROC\ MIXED}, \texttt{lmer}, \texttt{lavaan},
\texttt{Mplus}, \texttt{stan}, and \texttt{JAGS}. In the Appendix, we
complete specification of the model as a hierarchical Bayesian model and
use the \texttt{R}-package \texttt{BayesFactor} for analysis. We have
also made our analysis code available at
\url{https://github.com/PerceptionAndCognitionLab/ctx-reliability}.

Figure~\ref{fig:oneBlockEst}A and B shows the comparison for a single
block of Stroop data in Hedge et al. (2018). In this block there are
about \(L=42\) trials per individual per condition. There are two
panels---one for sample effects (\(d_i\)), and one for model-based
estimates of \(\theta_i\). Participants are ordered from the ones that
show the smallest Stroop effect to the ones that show the largest. In
the left panel, the solid line shows the sample effect for each
participant. The shaded area is the extent of each individual's 95\%
confidence interval. In the right panel, the solid line shows the point
estimate of \(\theta_i\) and the shaded area is the analogous 95\%
credible interval. The three horizontal lines are the mean (solid) and
one standard deviation markers (dashed).

\begin{figure}[htbp]
\centering
\includegraphics{p_files/figure-latex/oneBlockEst-1.pdf}
\caption{\label{fig:oneBlockEst}Analysis of a single A. Sample effects from
one block of data for all individuals with 95\% CIs. B. Model estimates
from one block of data for all individuals with 95\% credible intervals.
C. Hierarchical shrinkage in practice. The bottom row shows sample
estimates from one block; the middle row shows the same from the
hierarchical model; the top row shows sample estimates from all the
blocks. The shrinkage from the hierarchical model attenuates the
variability so that it matches that from much larger samples.}
\end{figure}

Although the grand mean effect is about the same for model estimates and
sample effects, individuals' effects are different. The sample effects
are spread out further from the grand mean than are the model estimates.
This increased spread is expected and, as discussed above, comes about
because the variability of \(d_i\) is the sum of true across-individual
variation (\(\sigma^2_\theta\)) and trial-by-trial variation
(\(2\sigma^2/L\)). The spread of model based estimates, in contrast,
reflects true individual variation alone. This reduction of variability
in hierarchical models is known as \emph{shrinkage} or
\emph{regularization} (Efron \& Morris, 1977). Regularization of
estimates is a well-known property of hierarchical models, and
regularized estimates are often more accurate than sample statistics
(Lehmann \& Casella, 1998).

Because the benefits of regularization remain opaque in many
hierarchical applications in cognitive psychology, we explore them in a
bit more depth here. The estimates in Figure~\ref{fig:oneBlockEst}B were
from a single block. Hedge et al. ran 10 such blocks distributed across
two sessions, and we may ask how well the estimates from one of the
blocks predict the whole 10-block set. The 10-block set consists of over
400 trials per person and condition. Figure~\ref{fig:oneBlockEst}C, the
style of which is borrowed from Efron and Morris (1977), shows the
sample estimates from one block (bottom row), the regularized model
estimates from the same block (middle row), and the sample estimates
from all ten blocks (top row). The shrinkage toward the grand mean in
hierarchical models is evident, and it brings the variability of the
model-based estimates from one block in line with the variability from
all ten blocks.

The model-based one-block estimators are better predictors of the
overall data than are the sample one-block estimators. This improvement
may be formalized by a root-mean-squared error measure. The error is
about 1.62 times larger for the sample means than for the hierarchical
model estimates. This benefit is general---hierarchical models provide
for more accurate estimates of individuals' latent abilities (James \&
Stein, 1961).

\section{A Two-Task Model for Reliability and
Correlation}\label{a-two-task-model-for-reliability-and-correlation}

The hierarchical model may be expanded from accounting for only one task
to account for two tasks or two sessions simultaneously. For two
sessions, the goal is to estimate a portable measure of test-retest
reliability; for two tasks, the goal is to measure a portable estimate
of correlation. Let's take reliability first. The Hedge et al. data set,
which we highlight here, has a novel feature. These researchers sought
to measure the test-retest reliability of several cognitive tasks. They
had individuals perform 720 trials of a task one day, and three weeks
later the individuals returned and performed another 720 trials. We let
the subscript \(j=1,2\) index the session. The trial-level model is
expanded to: \[
Y_{ijk\ell} \sim \mbox{Normal}(\alpha_{ij}+x_k\theta_{ij},\sigma^2).
\] Here, we simply expand the parameters to hold for
individual-by-session combinations.

The parameters of interest are \(\theta_{ij}\) and there are several
specifications that may be made here. We start with the most general of
these, an additive-components decomposition into common and opposite
components: \[
\begin{aligned}
\theta_{i1} &= \nu_1 + \omega_i - \gamma_i,\\
\theta_{i2} &= \nu_2 + \omega_i + \gamma_i.
\end{aligned}
\] The parameter \(\nu_j\) is the main effect of the \(j\)th session,
and by having separate main effect parameters for each session, the
model captures the possibility of a systematic effect of session on the
Stroop effect. The parameter \(\omega_i\) is the common effect of the
\(i\)th individual; individuals that have large Stroop effects on both
sessions have high values of \(\omega\). The parameter, \(\gamma_i\), is
the oppositional component. It captures idiosyncratic deviations where
one individual may have a large Stroop effect in one session and a
smaller one in another. These individual common effects and oppositional
effects are random effects, and we place a hierarchical constraint on
them: \[
\begin{aligned}
\omega_i &\sim \mbox{Normal}(0,\sigma^2_\omega),\\
\gamma_i &\sim \mbox{Normal}(0,\sigma^2_\gamma).
\end{aligned}
\]

To gain an expression for the correlation among two sessions, it is
helpful to express the multivariate distribution of the \(\theta\)s. The
easiest way to do this is to write out the distribution for two
individuals' effects across two sessions: \[
\begin{bmatrix} \theta_{11} \\ \theta_{12} \\ \theta_{21} \\ \theta_{22} \end{bmatrix}
\sim \mbox{N}_4\left(
\begin{bmatrix} \nu_1 \\ \nu_2 \\ \nu_1 \\ \nu_2 \end{bmatrix},
\begin{bmatrix} 
\sigma^2_\omega+\sigma^2_\gamma & \sigma^2_\omega-\sigma^2_\gamma& 0 & 0\\ 
\sigma^2_\omega-\sigma^2_\gamma & \sigma^2_\omega+\sigma^2_\gamma & 0 & 0\\
0 & 0 &\sigma^2_\omega+\sigma^2_\gamma & \sigma^2_\omega-\sigma^2_\gamma \\ 
0 & 0 & \sigma^2_\omega-\sigma^2_\gamma & \sigma^2_\omega+\sigma^2_\gamma
\end{bmatrix}
\right)_.
\] From these distributions, it follows that the test-retest
reliability, the correlation across \(\theta_{i1}\) and \(\theta_{i2}\),
is \[
\rho = \frac{\sigma^2_\omega-\sigma^2_\gamma}{\sigma^2_\omega+\sigma^2_\gamma}
\] Importantly, this quantity is portable as it does not include
trial-by-trial variability.

For comparison, we may also express in multivariate distribution for
sample effects for two individuals in two sessions: \[
\begin{bmatrix} d_{11} \\ d_{12} \\ d_{21} \\ d_{22} \end{bmatrix}
\sim \mbox{N}_4\left(
\begin{bmatrix} \nu_1 \\ \nu_2 \\ \nu_1 \\ \nu_2 \end{bmatrix},
\begin{bmatrix} 
\sigma^2_\omega+\sigma^2_\gamma +2\sigma^2/L& \sigma^2_\omega-\sigma^2_\gamma & 0 & 0\\ 
\sigma^2_\omega-\sigma^2_\gamma & \sigma^2_\omega+\sigma^2_\gamma + 2\sigma^2/L& 0 & 0\\
0 & 0 &\sigma^2_\omega+\sigma^2_\gamma + 2\sigma^2/L& \sigma^2_\omega-\sigma^2_\gamma \\ 
0 & 0 & \sigma^2_\omega-\sigma^2_\gamma & \sigma^2_\omega+\sigma^2_\gamma + 2\sigma^2/L
\end{bmatrix}
\right)_.
\] From this distribution, it is clear that the sample correlation
between sample effects is estimating
\((\sigma^2_\omega-\sigma_\gamma)/(\sigma^2_\omega+\sigma^2_\gamma+2\sigma^2/L)\).
It is the added sample noise in the denominator, \(\sigma^2/L\), that
renders the sample test-retest correlation importable and too small.

\begin{figure}[htbp]
\centering
\includegraphics{p_files/figure-latex/reliability-1.pdf}
\caption{\label{fig:reliability}Test-retest reliability for Hedge et al.'s
Stroop-task data set. The top row shows analysis for one block per
session; the bottom row shows analysis for all blocks. The model
analyses show substantial correlation even with one block. When all the
data are considered the test-retest correlation is well localized for
satisfactorily high values.}
\end{figure}

Figure~\ref{fig:reliability} provides a real-data comparison of
model-based and sample correlations. The top row is for the single block
of Hedge et al.'s Stroop task; the bottom row is for all ten blocks of
the same data set. The first column are scatter plots of the sample
effects of the first session vs.~the second session. There is almost no
test-retest correlation using a single block of data (A); but when all
data are considered there is a moderate correlation (B). This pattern is
the same as in Figure~\ref{fig:portability} where reliability was
diminished for smaller numbers of trials. The model estimates of
individual effects are shown in the middle column (C, D). Here, there is
shrinkage, especially for the single block (C). This shrinkage, however,
is to a line rather than to a point in the center. This pattern reflects
the model specification where positive and negative correlations are
explicitly modeled. The last column shows the posterior distributions of
the test-retest reliability coefficient. Here, for the single-block data
there is much uncertainty (E). The uncertainty shrinks as the sample
sizes grow, as indicated by the same plot for all the data (F). The
posterior mean, a point-estimate for the test-retest reliability of the
Stroop task, is 0.72. We also analyzed the flanker task, and the
test-retest reliability of this this task is 0.68.

Figure~\ref{fig:reliability} highlights the dramatic difference between
conventional and hierarchical-model analysis, especially for smaller
numbers of trials per participant. If one uses the conventional
correlation coefficient for one block of data, the resulting value is
small, 0.10, and somewhat well localized (the 95\% confidence interval
is from-0.17 to 0.36). We emphasize this result is misguided. When all
the data are used, the conventional correlation coefficient is 0.55,
which is well outside this 95\% confidence interval. Contrast this
result to that from the hierarchical model. Here, not only is the
correlation coefficient larger, 0.31, but the uncertainty is quite large
as well (the 95\% credible interval is from -0.32 to 0.82). Moreover,
the value of the correlation from the hierarchical model with all of the
data, 0.72, is within this credible interval. In summary, using the
conventional analysis results in over confidence in a wrong answer. The
hierarchical model, however, tempers this overconfidence by accounting
for trial-by-trial uncertainty as well as uncertainty across
individuals.

The above real-world demonstration that sample estimates of correlations
are so badly attenuated should be alarming. The one-block case was based
on a design with 48 trials per individual per condition. This is a
typical number for individual-difference batteries with a large number
of tasks. Yet, the attenuation is so severe that test-retest
correlations are barely detectable. No wonder observed correlations
across tasks are so low. The hierarchical model provides a more
sensitive view of the structure in the data by allowing for shrinkage to
a regression line.

\begin{table}[tbp]
\begin{center}
\begin{threeparttable}
\caption{\label{tab:corTab}Test-retest correlations for Hedge et al (2018) data sets.}
\begin{tabular}{llll}
\toprule
 & \multicolumn{1}{c}{Sample} & \multicolumn{1}{c}{Model} & \multicolumn{1}{c}{\thead{Corrected for \\ Attenuation}}\\
\midrule
Stroop, one block & 0.10 & 0.31 & 0.31\\
Stroop, full set & 0.55 & 0.72 & 0.74\\
Flanker, one block & 0.23 & 0.49 & 0.52\\
Flanker, full set & 0.50 & 0.68 & 0.71\\
\bottomrule
\end{tabular}
\end{threeparttable}
\end{center}
\end{table}

The attenuation of correlation from measurement error is well known in
classical test theory and one approach is to apply a correction formula.
The basic idea underlies the Spearman-Brown prophecy formula and the
Spearman formula for disattenuation of correlations (Spearman, 1904b).
We used the latter to compute an adjusted test-retest correlations for
both tasks and for both the one-block and full data sets. The results
are in Table~\ref{tab:corTab}, and there is a high degree of concordance
for the model correlations and the corrected correlation. This overall
concordance is expected as the correction for attenuation and the
hierarchical models share the same foundational assumption about noise.

Correlation between tasks may be handled with the same machinery. Here
we compare Stroop task performance to flanker task performance. Each of
Hedge et al's participants participated in both tasks. To correlate
tasks, we combined data across sessions and fit the model where \(j\)
indexes task rather than session. The results are shown in
Figure~\ref{fig:combo}. As can be seen, there appears to be no
correlation. Inference about the lack of correlation will be made in the
next section where model comparison is discussed.

\begin{figure}[htbp]
\centering
\includegraphics{p_files/figure-latex/combo-1.pdf}
\caption{\label{fig:combo}The lack of correlation between flanker task and
Stroop task performance. Left: Scatter plot of individuals' model-based
estimates. Right: Posterior distribution of the correlation
coefficient.}
\end{figure}

\section{Model Comparison}\label{model-comparison}

The above analyses were focused on parameter estimation. Model-based
estimation provided here are portable analogs to sample-based measures
of effect size, reliability, and correlation. The difference is that
they account for variation at the trial level, and consequently, may be
ported to designs with varying numbers of trials.

Researchers, however, are often interested in stating evidence for
theoretically meaningful propositions. In the next section, we describe
a set of theoretically meaningful propositions and their model
implementation. Following this, we present a Bayes factor method for
model comparison.

\subsection{Theoretical Positions and Model
Implementation}\label{theoretical-positions-and-model-implementation}

When assessing the relationship between two tasks, the main target is
the true latent correlation in the large-trial limit. There are two
opposing theoretically important positions: 1. that there is no
correlation; and 2. there is full correlation. A lack of true
correlation indicates that the two tasks are measuring independent
psychological processes or abilities. Likewise, if there is full
correlation, then the two tasks are measuring the same psychological
processes or abilities.

In the preceding section, we presented an estimation model, which we now
call the \emph{general model}. The critical specification is that of
\(\theta_{ij}\), the individual-by-task effect. We modeled these as:
\[ \begin{aligned}
\calM_g: \quad & \theta_{ij}=\nu_j+\omega_i+u_j\gamma_i,\\
& \omega_i \sim \mbox{Normal}(0,\sigma^2_\omega),\\
& \gamma_i \sim \mbox{Normal}(0,\sigma^2_\gamma),
\end{aligned}
\] where \(u=(-1,1)\) for the two tasks. In this model, the correlation
among an individuals reflects the variability of \(\omega\) and
\(\gamma\). All values of correlation on the open interval (-1,1) are
possible. Full correlation is not possible, and there is no special
credence given to no correlation. To represent the two theoretical
positions we develop alternative models on \(\theta{ij}\).

A no-correlation model is given by putting uncorrelated noise on
\(\theta_{ij}\): \[
\begin{aligned}
\calM_0: \quad & \theta_{ij} \sim \mbox{Normal}(\nu_j,\sigma^2_\theta).
\end{aligned}\]

The no-correlation and the general models provide for different
constraints. The general model has regularization to a regression line
reflected by the balance of the variabilities of \(\omega\) and
\(\gamma\). The no-correlation has regularization to the point
\((\nu_1,\nu_2)\).

A full correlation model is given by simply omitting the \(\gamma\)
parameters in the general model. \[
\begin{aligned}
\calM_1: \quad & \theta_{ij}=\nu_j+\omega_i,\\
& \omega_i \sim \mbox{Normal}(0,\sigma^2_\theta)
\end{aligned}
\] Here, there is a single random parameter, \(\omega_i\), for both
tasks per individual. In the full-correlation model, regularization is
to a line with a slope of 1.0.

\subsection{Bayes Factor Analysis}\label{bayes-factor-analysis}

We use the Bayes factor (Edwards, Lindman, \& Savage, 1963; Jeffreys,
1961) to measure the strength of evidence for the three models. The
Bayes factor is the probability of the observed data under one model
relative to the probability of the observed data under a competitor
model.

\begin{table}[tbp]
\begin{center}
\begin{threeparttable}
\caption{\label{tab:bftab}Bayes Factor Values for Competing Models of Correlation.}
\begin{tabular}{llll}
\toprule
 & \multicolumn{1}{c}{General} & \multicolumn{1}{c}{No Correlation} & \multicolumn{1}{c}{Full Correlation}\\
\midrule
Stroop & 1-to-1 & 1-to-2330 & 1-to-50\\
Flanker & 1-to-1 & 1-to-469 & 1-to-31\\
Stroop v. Flanker & 1-to-5.4 & 1-to-1 & 1-to-1.41$\times 10^{37}$\\
High Correlation & 1-to-31 & 1-to-1.03$\times 10^{5}$ & 1-to-1\\
\bottomrule
\addlinespace
\end{tabular}
\begin{tablenotes}[para]
\textit{Note.} Bayes factors are relative to the preferred model.  These by convention have Bayes factors of 1-to-1.  The remaining factors describe how much worse a model fares.
\end{tablenotes}
\end{threeparttable}
\end{center}
\end{table}

Table~\ref{tab:bftab} shows the Bayes factor results for the Stroop and
flanker task data sets from Hedge et al. (2018). The top two rows are
for the Stroop and flanker data, and the correlation being tested is the
test-retest reliability. The posterior mean of the correlation
coefficients are 0.72 and 0.68, for the Stroop and flanker tasks,
respectively. The Bayes factors confirm that there is ample evidence
that the correlation is neither null nor full. Hence, we may conclude
that there is indeed some though not a lot of added variability between
the first and second sessions in these tasks. The next row shows the
correlation between the two tasks. Here, the posterior mean of the
correlation coefficient is -0.06 and the Bayes factors confirm that the
no-correlation model is preferred. The final row is a demonstration of
the utility of the approach for finding dimension reductions. Here, we
split the flanker task data in half by odd and even trials rather than
by sessions. We then submitted these two sets to the model, and
calculated the correlation. It was quite high of course, and the
posterior mean of the correlation was 0.82. The Bayes factor analysis
concurred, and the full-correlation model was favored by 31-to-1 over
the general model, the nearest competitor.

The Appendix provides the prior settings for the above analyses. It also
provides a series of alternative settings for assessing how sensitive
Bayes factors are to reasonable variation in priors. With these
alternative settings, the Bayes factors attain different values.
Table~\ref{tab:sens} in the Appendix shows the range of Bayes factors
corresponding to these alternative settings. This table provides context
for understanding the limits of the data and the diversity of opinion
they support.

\section{General Discussion}\label{general-discussion}

In this paper we examined classical test theory analysis of experimental
tasks. The main difficulty in classical analysis occurs when researchers
aggregate across trials to form individual-by-task scores. We recommend
that researchers avoid this aggregation. If aggregated scores are used
as input, then conventional sample measures are not interpretable
without correction because they estimate quantities contaminated by
removable trial-by-trial variation. With this contamination, effect
sizes, reliabilities, and correlations are too low, sometimes
dramatically so. We advocate applying hierarchical models to the trial
level to remove trial-by-trial variation. Concepts such as effect size,
reliability, and correlation are portable when defined in the asymptotic
limit of unbounded trials per individual. In the current models,
performance in these asymptotic limits are explicit model parameters.

With this development, it is possible to assess whether observed low
correlations across tasks reflect low reliability or a true lack of
association. We examine this problem for the Stroop and flanker task
data reported in Hedge et al. (2018). We find that there is relatively
high test-retest reliability for both tasks. This high reliability
allows for the interpretation of the correlation between Stroop and
flanker tasks. There is direct evidence for a null correlation.

Individual-difference researchers are intimately familiar with mixed
linear models, and these are used regularly to decompose variability.
Adding one additional level, the trial level, is conceptually trivial
and computationally straightforward. Indeed, modeling at this level is
common in high-stakes testing (IRT, Lord \& Novick, 1968), cognition
(Lee \& Webb, 2005; Rouder \& Lu, 2005), and linguistics (Baayen et al.,
2002). In the Appendix, we show how the \texttt{nWayAOV} function in the
\texttt{BayesFactor} package may be used for implementation.

\subsection{Models vs.~Corrections}\label{models-vs.corrections}

One of the promising results here is that existing corrections for
measurement noise yield similar results to hierarchical modeling. This
is not surprising as these corrections are based on the same assumptions
about noise. Yet, we believe that researchers should invest in modeling
even though correction formulas are much easier to implement in
practice. Here is why: First, correction yields a point estimate of the
true latent value without a sense of uncertainty. Model-based point
estimates, like the ones presented here, often come with a measure of
uncertainty. Second, models may be compared to answer questions about
whether the correlation is zero, one, or intermediary. Corrections offer
no such inferences. Third, models may be extended to represent more
complex structure in data. We know of no comparable correction for
measurement noise in say factor analysis. Fourth, and most importantly,
modeling provides for a deeper and more nuanced understanding of
structure in data. Knowing how models represent structure and how data
influences model-based conclusions allows researchers to add value in
addressing theoretically-important substantive issues in a way that
correction procedures cannot.

\subsection{More Advanced Task Models}\label{more-advanced-task-models}

The field of individual differences has moved far beyond the
consideration of two tasks or instruments. The field is dominated by
multivariate, latent-variable models including factor models, structural
equation models, state-space models, and growth models (e.g., Bollen,
1989). When task scores are used with these advanced models, the results
are importable and, consequently, difficult to interpret unless
trial-by-trial variation is modeled. A critical question is whether the
hierarchical models specified here extend well beyond two tasks. The
generalization, at least for estimation, is straightforward. Bayesian
development of latent variable models is a burgeoning field (Lee, 2007).
Moreover, because Bayesian analysis explicitly allows for conditional
rather than marginal expressions, adding covariance models is
conceptually straightforward. Computational issues have been well
explored, and general-purpose packages such as Stan (Carpenter et al.,
2017) and JAGS (Plummer, 2003) are well suited to developing advanced
latent variable models that account for trial-by-trial variation.

Although some forms of analysis are straightforward in Bayesian
latent-variable modeling, Bayes-factor model comparison is not among
these. Developing Bayes factors for complicated mixed models is
certainly timely and topical, but the computational issues may be
difficult. As a result, there is much work to be done if one wishes to
state the evidence for theoretically motivated constraints on
covariation across several tasks.

\subsection{A Caveat}\label{a-caveat}

In the beginning of this paper, we asked if low correlations reflect
statistical considerations or substantive considerations. The answer
here, with a large data set, is that there is a substantive claim. We
state evidence for a lack of correlation across Stroop and flanker
tasks. That said, we should not understate the statistical
considerations.

We suspect there is less true individual variability in many tasks than
has been realized, and apparent variability comes more from the trial
level than from true individual differences. Consider a typical priming
task. Trials in these tasks take about 500 ms to complete, and a large
effect is about 50 ms. If the average is 50 ms, how much could
individuals truly vary? The answer, we believe, is \enquote{not that
much,} especially if we assume that no individual has true negative
effects (Haaf \& Rouder, 2017; Rouder \& Haaf, 2018). Indeed, we have
analyzed the observed and true (latent) variation across individuals in
several tasks (Haaf \& Rouder, 2017, 2018). Observed variation is
usually on the order of 100s of milliseconds. Yet, once trial-by-trial
variation is modeled, individuals' true values vary from 10 ms to 40 ms.
For such a narrow range, accurate understanding of individuals would
require estimating each individual's effect to within a few
milliseconds. Even the hierarchical models we advocate cannot mitigate
this fact; if there is too little resolution then the posterior
reliabilities and correlations will not be well localized. Obtaining the
requisite level of resolution to study individual differences in these
tasks may requires several hundred trials per individual per condition,
and perhaps this design choice may prove more critical than collecting
from hundreds of individuals.

\newpage

\section{Appendix}\label{appendix}

\subsection{Model Specification}\label{model-specification-1}

\subsubsection{The One-Task Model.}\label{the-one-task-model.}

The model of data for a single task is given by: \[
Y_{ik\ell}|\alpha_i,\theta_i,\sigma^2 \stackrel{\mbox{ind}}{\sim }\mbox{Normal}(\alpha_{i}+x_k\theta_{i},\sigma^2),
\] where \(i\), \(k\), and \(\ell\) respectively index individual,
condition, and replicate, and and where \(x_1=-1/2\), and \(x_2=1/2\).
Models on \(\alpha_i\) and \(\theta_i\) are \[
\begin{aligned}
\alpha_i|\mu_\alpha,\sigma^2_\alpha &\stackrel{\mbox{iid}}{\sim} \mbox{Normal}(\mu_\alpha,\sigma^2_\alpha),\\
\theta_i|\mu_\theta,\sigma^2_\theta &\stackrel{\mbox{iid}}{\sim} \mbox{Normal}(\mu_\theta,\sigma^2_\theta),
\end{aligned}
\]

We use a \(g\)-prior specification for variance components (Zellner \&
Siow, 1980,Zellner (1986)), These priors have become popular for linear
models {[}Liang, Paulo, Molina, Clyde, and Berger
(2008),Rouder:etal:2012a{]}. Let \(g_\alpha=\sigma^2_\alpha/\sigma^2\)
and \(g_\theta=\sigma^2_\theta/\sigma^2\). Then: \[
\pi(\mu_\alpha,\sigma^2) \propto 1/\sigma^2,
\] and \[
\begin{aligned}
\mu_\theta & \sim \mbox{Normal}(0,g_{\mu_\theta}\sigma^2),\\
g_\alpha & \sim \mbox{inverse-$\chi^2$}(1,r^2_\alpha),\\
g_\theta & \sim \mbox{inverse-$\chi^2$}(1,r^2_\theta),\\
g_{\mu_\theta} & \sim \mbox{inverse-$\chi^2$}(1,r^2_{\mu_\theta}),
\end{aligned}
\] where inverse-\(\chi^2\)(a,b) is a scaled inverse chi-squared
distribution with \(a\) degrees-of-freedom and a scale of \(b\) (see
Gelman, Carlin, Stern, \& Rubin, 2004). The inverse-\(\chi^2\) prior on
\(g\) is a popular choice because it is computationally convenient and
leads to posteriors with desirable objective Bayes properties {[}Berger
(2006)\href{mailto:;@Liang}{\nolinkurl{;@Liang}}:etal:2008;Rouder:etal:2012{]}.

\subsubsection{The Two-Task Model}\label{the-two-task-model}

The model of data for two task is given by: \[
Y_{ik\ell}|\alpha_{ij},\theta_{ij},\sigma^2 \stackrel{\mbox{ind}}{\sim }\mbox{Normal}(\alpha_{ij}+x_k\theta_{ij},\sigma^2),
\] where \(j\) indexes the task. Models on \(\alpha_{ij}\) and
\(\theta_{ij}\) are \[
\alpha_{ij} \stackrel{\mbox{iid}}{\sim} \mbox{Normal}(\mu_\alpha,g_\alpha\sigma^2),\]
and \[
\begin{aligned}
\theta_{i1}|\nu_1,\omega_i,\gamma_i &= \nu_1 + \omega_i - \gamma_i,\\
\theta_{i2}|\nu_2,\omega_i,\gamma_i &= \nu_2 + \omega_i + \gamma_i.
\end{aligned}
\] Models on \(\nu_k\), \(\omega_i\) and \(\gamma_i\) are \[
\begin{aligned}
\nu_k|g_\nu,\sigma^2 & \stackrel{\mbox{iid}}{\sim}\mbox{Normal}(0,g_\nu\sigma^2)\\
\omega_i|g_\omega\sigma^2 &\stackrel{\mbox{iid}}{\sim} \mbox{Normal}(0,g_\omega\sigma^2),\\
\gamma_i|g_\gamma,\sigma^2 &\stackrel{\mbox{iid}}{\sim} \mbox{Normal}(0,g_\gamma\sigma^2).
\end{aligned}
\] Priors on \[
\begin{aligned}
g_\nu &\sim \mbox{inverse-$\chi^2$}(1,r^2_\nu),\\
g_\omega &\sim \mbox{inverse-$\chi^2$}(1,r^2_\omega),\\
g_\gamma &\sim \mbox{inverse-$\chi^2$}(1,r^2_\gamma),
\end{aligned}
\]

\subsection{Prior settings}\label{prior-settings}

The above models are estimated in the Bayesian framework, and as such,
priors are placed on parameters. The critical settings are for the scale
parameters \(r_\alpha\), \(r_\theta\), \(r_{\mu_theta}\), \(r_\nu\),
\(r_\omega\), and \(r_\gamma\) In this section, we justify our choices
for these settings. In the next section, we discuss how the choices
affect analysis.

As substantive scientist we have background knowledge of how variable
effects tend to be in comparable experiments. In a typical Stroop or
flanker experiment, the trial-to-trial standard deviation is somewhere
around 200 ms to 300 ms. Hedge et al. (2018) had particularly fast and
error prone responses in their data set, and as a result, we take the
200 ms value to help set prior scale constants. In our experience people
also vary about this much, that is, in ordinary response time
experiments, that variation of people is on the order of a few hundred
milliseconds. We set \(r_\alpha=\) 1 to reflect that the expected
trial-to-trial variation and expected variation across people on over
response time is about the same size.

A healthy Stroop or flanker effect here would be about 50 ms, or 0.25 of
the trial-to-trial standard deviation. Hence, we set The next question
is how much do we think people could possible vary in each task. We
chose 33.33 ms or \(r_\theta=\) 0.17 of the trial-to-trial standard
deviation, and this value was used in both the full correlation and no
correlation models. This value reflects our belief that nobody has
negative true Stroop of flanker effects, and therefore, the overall size
of the effect, 50 ms, limits the possible size of the variability. To
set \(r_\omega\) and \(r_\gamma\), we equated the bivariate variances
between the general model and no correlation model and chose
\(r_\gamma\) to be 0.67 of \(r_\omega\). The reason we chose this ratio
is that we \emph{a priori} expect some positive covariation across the
tasks. With these choices, the value of \(r_\omega\)=0.14 and the value
of \(r_\gamma\)=0.10.

Figure~\ref{fig:prior}A shows the prior on standard deviations
\(\sigma_\theta\), \(\sigma_\omega\) and \(\sigma_\gamma\) based on
these choices. As can be seen, although the priors have scales, their
flexibility comes from their slow, fat right tails.

The next panel, Figure~\ref{fig:prior}B shows how these choices specify
a prior over correlation, \(\rho\), in the general model. The
distribution is \(u\)-shaped which is reasonable for priors on bounded
spaces (Jaynes, 1986). The slight weight toward positive correlations
reflects the choice that \(r_\omega>r_\gamma\). Had these two settings
been equal, then there would be no slight weight toward positive and
negative values. If \(r_\omega<r_\gamma\), the weight is toward negative
values.

\begin{figure}[htbp]
\centering
\includegraphics{p_files/figure-latex/prior-1.pdf}
\caption{\label{fig:prior}Prior specifications. A. The critical priors are
on \(\sigma^2_\theta\), \(\sigma^2_\omega\), and \(\sigma^2_\gamma\).
Shown are the density functions for the standard deviations
(\(\sigma_\theta, \sigma\omega, \sigma_\gamma\)). B. The implied prior
for the correlation coefficient \(\rho\).}
\end{figure}

\subsection{How To Analyse The Model}\label{how-to-analyse-the-model}

We analyzed the above models in the \texttt{BayesFactor} package (Morey
\& Rouder, 2015) in the \texttt{R} environment. This package provides
for the analysis of a wide variety of ANOVA and regression models. It
has been a key infrastructure component of the increasing popularity of
Bayes factor model comparison in psychological science, and serves as
the back end computational engine in JASP (Love et al., 2015). One of
the key advantages here is that \texttt{BayeFactor} implements the
\(g\)-prior set up, and this fact makes using it simple. The critical
function for analysis is the function \texttt{nWayAOV()} (documentation
available at
\url{https://www.rdocumentation.org/packages/BayesFactor/versions/0.9.12-4.2/topics/nWayAOV})
This function outputs posterior distributions for all parameters and a
Bayes factor model comparison statistic. The inputs to it are the data,
the models defined by a design matrix, and the \(g\)-prior scale
settings.

As mentioned previously, this paper and all the code needed for analysis
and graphs are available at
\url{https://github.com/PerceptionAndCognitionLab/ctx-reliability}. The
heart of our code is at
\href{https://github.com/PerceptionAndCognitionLab/ctx-reliability/blob/public/share/lib.R}{share/lib.R}.
The \texttt{R} script contains functions for loading Hedge et al's data
and cleaning it. The function \texttt{design1Session} sets up a design
matrix for the single-task model for inputted data. The function
\texttt{est1Session} then returns the model-based estimates for the
specified prior settings. The functions for two tasks work similarly.

\subsection{Sensitivity to Prior
Specification}\label{sensitivity-to-prior-specification}

When performing parameter estimation, the above models are rather robust
to the choice of prior form and prior settings. The reason is simple,
the priors are rather diffuse and the sample sizes in the data sets are
quite large. Changes in the scale settings has minimal effects on the
posterior distributions of the parameters.

The influence of the prior settings is more relevant for model
comparison, and the Bayes factor values are dependent on prior
specification. A few points of context are helpful in understanding this
dependence. It seems reasonable to expect that if two researchers run
the same experiment and obtain the same data, then they should reach
similar conclusions. To meet this expectation, many Bayesian analysts
actively seek to minimize this dependence by picking likelihoods, prior
parametric forms, and heuristic methods of inference so that variation
in prior settings have minimal influence (Aitkin, 1991; Gelman \&
Shalizi, 2013; Kruschke, 2014; Spiegelhalter, Best, Carlin, \& Linde,
2002). In the context of these views, the dependence of prior settings
on inference is viewed negatively; not only is it something to be
avoided, it is a threat to the validity of Bayesian analysis.

We reject this expectation that minimization of prior effects is
necessary or even laudable. Rouder, Morey, and Wagenmakers (2016) argue
that the goal of analysis is to add value by searching for
theoretically-meaningful structure in data. Vanpaemel (2010) and
Vanpaemel and Lee (2012) provide a particularly appealing view of the
prior in this light. Accordingly, the prior is where theoretically
important constraint is encoded in the model. When different researchers
use different priors, they are testing different theoretical
constraints, and not surprisingly, they will reach different opinions
about the data. Rouder et al. (2016) argue that this variation is not
problematic in fact it should be expected and seen positively. Methods
that are insensitive to different theoretical commitments are not very
useful. Rouder et al. (2016) recommend that so long as various prior
settings are justifiable, the variation in results should be embraced as
the legitimate diversity of opinion. When reasonable prior settings
result in conflicting conclusions, we realize the data do not afford the
precision to adjudicate among the positions.

The critical prior specifications in the models are the settings
\(r_\theta\), \(r_\omega\), and \(r_\gamma\). How does the Bayes factor
change if we make other reasonable choices? Let's start by noting that
about a 50 ms effect is reasonably expected. We cannot imagine
individual variation being so large as to have a standard deviation much
greater than this value. If say the true value of \(\sigma_\theta\) was
50 ms, it would imply that 14\% of individuals have true negative Stroop
or flanker effects, which seems implausible. Likewise, we cant imagine
variation being smaller than say 10 ms across people. These values, 10
ms and 50 ms, inform a bracket of reasonable settings for \(r_\theta\).

The same approach works for finding reasonable ranges for the ratio of
\(r_\gamma/r_\omega\). The chosen value was 0.67 meaning that mildly
positive correlations were expected. We think a reasonable range for
this ratio is from a high value of 1.0 to a low value of 1/3. The value
of 1.0 corresponds to as great a chance of negative correlation as
positive, which is the bottom limit on the possible reasonable ranges of
correlation for tasks that purportedly tap the same underlying
construct. The value of 1/3 sets a lofty expectation of positive
correlation. We cannot imagine settings greater than or less than these
values would be reasonable for the general model.

Table~\ref{tab:sens} shows how variation in prior settings resulted
variation of the Bayes factors. The first three columns show the
settings for \(r_\theta\), \(r_\omega\), and \(r_\gamma\). The next two
columns show Bayes factors for the winning model for two cases. The
first is the correlation between the Stroop and flanker task, and the
Bayes factor is by how much the null-correlation model is preferred over
the general model. The second is the correlation among even and odd
trials in the flanker task, and the Bayes factor is by how much the
full-correlation model is preferred over the general model. As can be
see, the prior settings do matter. Take the correlation between the
Stroop and flanker tasks. The null correlation model wins in all cases,
but the value ranges from about 5-to-1 to about 11-to-1 over the general
model. These are relatively stable results bolstering the claim that
there is evidence for a lack of correlation. For the high reliability
case, the full correlation model wins in all cases, but the Bayes factor
values are quite variable, from 3-to-1 to 176-to-1. Here we are less
comfortable because the evidence is more dependent on prior assumptions.
Had we taken a larger bracket, the model preferences may have even
reversed. While we may favor a full correlation model, that preference
should be tempered by these sensitivity analyses.

\begin{table}[tbp]
\begin{center}
\begin{threeparttable}
\caption{\label{tab:sens}Sensitivity to Prior Settings}
\begin{tabular}{lllcc}
\toprule
$r_\theta$ & \multicolumn{1}{c}{$r_\omega$} & \multicolumn{1}{c}{$r_\gamma$} & \multicolumn{1}{c}{Stroop v. Flanker, $B_{0g}$} & \multicolumn{1}{c}{High Correlation, $B_{1g}$}\\
\midrule
0.17 & 0.14 & 0.1 & 5.4 & 31\\
0.25 & 0.22 & 0.14 & 8.6 & 170\\
0.05 & 0.04 & 0.03 & 9.3 & 2.9\\
0.17 & 0.12 & 0.12 & 4.8 & 76\\
0.17 & 0.2 & 0.07 & 11 & 12\\
\bottomrule
\addlinespace
\end{tabular}
\begin{tablenotes}[para]
\textit{Note.} $B_{0g}=$ Support for Null Model over General Model; $B_{1g}=$ Support for Full Model over General Model.
\end{tablenotes}
\end{threeparttable}
\end{center}
\end{table}

\newpage

\section{References}\label{references}

\begingroup
\setlength{\parindent}{-0.5in} \setlength{\leftskip}{0.5in}

\hypertarget{refs}{}
\hypertarget{ref-Aitkin:1991}{}
Aitkin, M. (1991). Posterior Bayes factors. \emph{Journal of the Royal
Statistical Society. Series B (Methodological)}, \emph{53}(1), 111--142.
Retrieved from \url{http://www.jstor.org/stable/2345730}

\hypertarget{ref-Baayen:etal:2002}{}
Baayen, R. H., Tweedie, F. J., \& Schreuder, R. (2002). The subjects as
a simple random effect fallacy: Subject variability and morphological
family effects in the mental lexicon. \emph{Brain and Language},
\emph{81}, 55--65.

\hypertarget{ref-Berger:2006}{}
Berger, J. O. (2006). The case for objective Bayesian analysis.
\emph{Bayesian Analysis}, \emph{1}, 385--402.

\hypertarget{ref-Bollen:1989}{}
Bollen, K. A. (1989). \emph{Structural equations with latent variables}.
Wiley.

\hypertarget{ref-Carpenter:etal:2017}{}
Carpenter, B., Gelman, A., Hoffman, M. D., Lee, D., Goodrich, B.,
Bettencourt, M., \ldots{} Riddell, A. (2017). Stan: A probabilistic
programming language. \emph{Journal of Statistical Software}, \emph{76}.

\hypertarget{ref-Edwards:etal:1963}{}
Edwards, W., Lindman, H., \& Savage, L. J. (1963). Bayesian statistical
inference for psychological research. \emph{Psychological Review},
\emph{70}, 193--242. Retrieved from
\url{http://dx.doi.org/10.1037/h0044139}

\hypertarget{ref-Efron:Morris:1977}{}
Efron, B., \& Morris, C. (1977). Stein's paradox in statistics.
\emph{Scientific American}, \emph{236}, 119--127.

\hypertarget{ref-Eriksen:Eriksen:1974}{}
Eriksen, B. A., \& Eriksen, C. W. (1974). Effects of noise letters upon
the identification of a target letter in a nonsearch task.
\emph{Perception \& Psychophysics}, \emph{16}, 143--149.

\hypertarget{ref-Friedman:Miyake:2004}{}
Friedman, N. P., \& Miyake, A. (2004). The relations among inhibition
and interference control functions: A latent-variable analysis.
\emph{Journal of Experimental Psychology: General}, \emph{133},
101--135.

\hypertarget{ref-Gelman:Shalizi:2013}{}
Gelman, A., \& Shalizi, C. R. (2013). Philosophy and the practice of
Bayesian statistics. \emph{British Journal of Mathematical and
Statistical Psychology}, \emph{66}, 57--64.

\hypertarget{ref-Gelman:etal:2004}{}
Gelman, A., Carlin, J. B., Stern, H. S., \& Rubin, D. B. (2004).
\emph{Bayesian data analysis (2nd edition)}. London: Chapman; Hall.

\hypertarget{ref-Green:etal:2016}{}
Green, S. B., Yang, Y., Alt, M., Brinkley, S., Gray, S., Hogan, T., \&
Cowan, N. (2016). Use of internal consistency coefficients for
estimating reliability of experimental task scores. \emph{Psychonomic
Bulletin \& Review}, \emph{23}(3), 750--763.

\hypertarget{ref-Haaf:Rouder:2017}{}
Haaf, J. M., \& Rouder, J. N. (2017). Developing constraint in Bayesian
mixed models. \emph{Psychological Methods}, \emph{22}(4), 779--798.

\hypertarget{ref-Haaf:Rouder:inpress}{}
Haaf, J. M., \& Rouder, J. N. (2018). Some do and some don't? Accounting
for variability of individual difference structures. \emph{Psychonomic
Bulletin and Review}. Retrieved from \url{https://psyarxiv.com/zwjtp/}

\hypertarget{ref-Hedge:etal:2018}{}
Hedge, C., Powell, G., \& Sumner, P. (2018). The reliability paradox:
Why robust cognitive tasks do not produce reliable individual
differences. \emph{Behavioral Research Methods}.

\hypertarget{ref-Ito:etal:2015}{}
Ito, T. A., Friedman, N. P., Bartholow, B. D., Correll, J., Loersch, C.,
Altamirano, L. J., \& Miyake, A. (2015). Toward a comprehensive
understanding of executive cognitive function in implicit racial bias.
\emph{Journal of Personality and Social Psychology}, \emph{108}(2), 187.

\hypertarget{ref-James:Stein:1961}{}
James, W., \& Stein, C. (1961). Estimation with quadratic loss. In
\emph{Proceedings of the fourth berkeley symposium on mathematical
statistics and probability} (pp. 361--379 ,).

\hypertarget{ref-Jaynes:1986}{}
Jaynes, E. (1986). Bayesian methods: General background. In J. Justice
(Ed.), \emph{Maximum-entropy and bayesian methods in applied
statistics}. Cambridge: Cambridge University Press.

\hypertarget{ref-Jeffreys:1961}{}
Jeffreys, H. (1961). \emph{Theory of probability (3rd edition)}. New
York: Oxford University Press.

\hypertarget{ref-Kruschke:2014}{}
Kruschke, J. K. (2014). \emph{Doing bayesian data analysis, 2nd edition:
A tutorial with r, jags, and stan}. Waltham, MA: Academic Press.

\hypertarget{ref-Lee:Webb:2005}{}
Lee, M. D., \& Webb, M. R. (2005). Modeling individual differences in
cognition. \emph{Psychonomic Bulletin \& Review}, \emph{12}(4),
605--621.

\hypertarget{ref-Lee:2007}{}
Lee, S.-Y. (2007). \emph{Structural equation modelling: A bayesian
approach.} New York: Wiley.

\hypertarget{ref-Lehmann:Casella:1998}{}
Lehmann, E. L., \& Casella, G. (1998). \emph{Theory of point estimation,
2nd edition}. New York: Springer.

\hypertarget{ref-Liang:etal:2008}{}
Liang, F., Paulo, R., Molina, G., Clyde, M. A., \& Berger, J. O. (2008).
Mixtures of g-priors for Bayesian variable selection. \emph{Journal of
the American Statistical Association}, \emph{103}, 410--423. Retrieved
from \url{http://pubs.amstat.org/doi/pdf/10.1198/016214507000001337}

\hypertarget{ref-Lord:Novick:1968}{}
Lord, F. M., \& Novick, M. R. (1968). \emph{Statistical theories of
mental test scores}. Reading, MA: Addison-Wesley.

\hypertarget{ref-Love:etal:2015}{}
Love, J., Selker, R., Verhagen, J., Smira, M., Wild, A., Marsman, M.,
\ldots{} Wagenmakers, E.-J. (2015). Software to sharpen your stats.
\emph{APS Observer}, \emph{28}, 27--29.

\hypertarget{ref-MacLeod:1991}{}
MacLeod, C. (1991). Half a century of research on the Stroop effect: An
integrative review. \emph{Psychological Bulletin}, \emph{109}, 163--203.

\hypertarget{ref-Morey:Rouder:BayesFactorPackage}{}
Morey, R. D., \& Rouder, J. N. (2015). BayesFactor 0.9.12-2.
Comprehensive R Archive Network. Retrieved from
\url{http://cran.r-project.org/web/packages/BayesFactor/index.html}

\hypertarget{ref-Pettigrew:Martin:2014}{}
Pettigrew, C., \& Martin, R. C. (2014). Cognitive declines in healthy
aging: Evidence from multiple aspects of interference resolution.
\emph{Psychology and Aging}, \emph{29}(2), 187.

\hypertarget{ref-Plummer:2003}{}
Plummer, M. (2003). JAGS: A program for analysis of Bayesian graphical
models using Gibbs sampling. In \emph{Proceedings of the 3rd
international workshop on distributed statistical computing}.

\hypertarget{ref-ReyMermet:etal:2018}{}
Rey-Mermet, A., Gade, M., \& Oberauer, K. (2018). Should we stop
thinking about inhibition? Searching for individual and age differences
in inhibition ability. \emph{Journal of Experimental Psychology:
Learning, Memory, and Cognition}. Retrieved from
\url{http://dx.doi.org/10.1037/xlm0000450}

\hypertarget{ref-Rouder:Haaf:2018a}{}
Rouder, J. N., \& Haaf, J. M. (2018). Power, dominance, and constraint:
A note on the appeal of different design traditions. \emph{Advances in
Methods and Practices in Psychological Science}, \emph{1}, 19--26.
Retrieved from \url{https://doi.org/10.1177/2515245917745058}

\hypertarget{ref-Rouder:Lu:2005}{}
Rouder, J. N., \& Lu, J. (2005). An introduction to Bayesian
hierarchical models with an application in the theory of signal
detection. \emph{Psychonomic Bulletin and Review}, \emph{12}, 573--604.

\hypertarget{ref-Rouder:etal:2016b}{}
Rouder, J. N., Morey, R. D., \& Wagenmakers, E.-J. (2016). The interplay
between subjectivity, statistical practice, and psychological science.
\emph{Collabra}, \emph{2}, 6. Retrieved from
\url{http://doi.org/10.1525/collabra.28}

\hypertarget{ref-Schmiedek:etal:2007}{}
Schmiedek, F., Oberauer, K., Wilhelm, O., Süß, H.-M., \& Wittmann, W. W.
(2007). Individual differences in components of reaction time
distributions and their relations to working memory and intelligence.
\emph{Journal of Experimental Psychology: General}, \emph{136}(3), 414.

\hypertarget{ref-Skrondal:Rabe-Hesketh:2004}{}
Skrondal, A., \& Rabe-Hesketh, S. (2004). \emph{Generalized latent
variable modeling: Multilevel, longitudinal, and structural equation
models}. Boca Raton: CRC Press.

\hypertarget{ref-Spearman:1904}{}
Spearman, C. (1904a). 'General intelligence,' obectively determined and
measured. \emph{American Journal of Psychology}, \emph{15}, 201--293.

\hypertarget{ref-Spearman:1904a}{}
Spearman, C. (1904b). The proof and measurement of association between
two things. \emph{American Journal of Psychology}, \emph{15}, 72--101.
Retrieved from
\url{https://www.jstor.org/stable/pdf/1412159.pdf?refreqid=excelsior\%3Af2a400c0643864ecfb26464f09f022ce}

\hypertarget{ref-Spiegelhalter:etal:2002}{}
Spiegelhalter, D. J., Best, N. G., Carlin, B. P., \& Linde, A. van der.
(2002). Bayesian measures of model complexity and fit (with discussion).
\emph{Journal of the Royal Statistical Society, Series B (Statistical
Methodology)}, \emph{64}, 583--639.

\hypertarget{ref-Stahl:etal:2014}{}
Stahl, C., Voss, A., Schmitz, F., Nuszbaum, M., Tüscher, O., Lieb, K.,
\& Klauer, K. C. (2014). Behavioral components of impulsivity.
\emph{Journal of Experimental Psychology: General}, \emph{143}(2), 850.

\hypertarget{ref-Stroop:1935}{}
Stroop, J. R. (1935). Studies of interference in serial verbal
reactions. \emph{Journal of Experimental Psychology}, \emph{18},
643--662.

\hypertarget{ref-Vanpaemel:2010}{}
Vanpaemel, W. (2010). Prior sensitivity in theory testing: An apologia
for the Bayes factor. \emph{Journal of Mathematical Psychology},
\emph{54}, 491--498.

\hypertarget{ref-Vanpaemel:Lee:2012}{}
Vanpaemel, W., \& Lee, M. D. (2012). Using priors to formalize theory:
Optimal attention and the generalized context model. \emph{Psychonomic
Bulletin \& Review}, \emph{19}, 1047--1056.

\hypertarget{ref-Voelkle:etal:2014}{}
Voelkle, M. C., Brose, A., Schmiedek, F., \& Lindenberger, U. (2014).
Toward a unified framework for the study of between-person and
within-person structures: Building a bridge between two research
paradigms. \emph{Multivariate Behavioral Research}, \emph{49}(3),
193--213.

\hypertarget{ref-Zellner:1986}{}
Zellner, A. (1986). On assessing prior distirbutions and Bayesian
regression analysis with \(g\)-prior distribution. In P. K. Goel \& A.
Zellner (Eds.), \emph{Bayesian inference and decision techniques: Essays
in honour of Bruno de Finetti} (pp. 233--243). Amsterdam: North Holland.

\hypertarget{ref-Zellner:Siow:1980}{}
Zellner, A., \& Siow, A. (1980). Posterior odds ratios for selected
regression hypotheses. In J. M. Bernardo, M. H. DeGroot, D. V. Lindley,
\& A. F. M. Smith (Eds.), \emph{Bayesian statistics: Proceedings of the
First International Meeting held in Valencia (Spain)} (pp. 585--603).
University of Valencia.

\endgroup






\end{document}
